Spectra resulting from measurements of Co-60, Am-241, Cs-137,and Ba-133 check
sources were provided to students via a publicly-accessible dropbox folder. Via
a python script, this data was downloaded and placed in NumPy arrays for
analysis. Peaks in the spectra corresponding to photopeaks of known energy were
then picked out manually and fitted using a gaussian distribution using SciPy's
curvefit tool. The estimates produced by curvefit for /mu (in units of channel number) for the
661.7keV and 59.5keV lines from the Cs and Am sources were then
used to develop a linear calibration of the form \[E_x = a*x + b\] (in which x denotes
the channel index).This calibration was then applied to the dataset gathered from
a Ba-133 check source in order to determine the accuracy of the fit.
