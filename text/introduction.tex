High purity germanium (HPGe) detectors offer a means of high energy resolution gamma
ray counting and are considered the gold standard for lab-scale gamma
spectroscopy(citation). These detectors require constant crogenic cooling when in use to
diminish thermal promotion of charge carriers to the conduction band{citation}. As a
result of thermal cycling and drift in readout electronics, these detectors must
be regularly energy calibrated. This is typically carried out using a number of
check sources which span the range of gamma ray energies one is likely to encounter
in everyday life (~.05-2MeV). \par
While calibration can theoretically be carried out using a minimum of two distinct
photopeaks, standard operating procedures in most cases call for at least three
of such peaks spaced widely across the useful energy range. This approach minimizes
error and provides information on the linearity of the detector's energy response. \par
In this lab, we process spectra from three distinct sources (cesium-137, americium-241
and barium-133) and use the channel numbers associated with the Cs and Am photopeaks
to develop a two-point energy calibration model. We then apply this calibration to the
barium-133 spectrum, which shows small errors (order of 1keV) in energy calibration in the
~300-500keV range. 
